\section{Methods}
\label{Methods}

Attempting to accurately forecast the fatality impact of COVID-19 requires understanding the underlying variables present in the disease.  These include the R\typesubscript{0}, the fatality rate, the amount of active cases, hospital utilization percentage, and the time delay between contracting the disease and death.  Using estimates for the fatality rate and time delay between contraction and demise, the amount of active cases can be interpolated.  As previously mentioned, due to lack of mass testing in the United States, relying on the actual data for total cases is unwise.  This total case data is largely just a function of total tests, rather than a true measure of the total active cases.  All daily case data was retrieved from Johns Hopkins University and Medicine Coronavirus Resource Center \cite{JHU}. While the "confirmed cases" data can be suspect, the death data is accurate and as immune to undertesting as it can be.  The only caveat being that fatalities that are not reported due to not going to the hospital are not captured.  The United States is currently at 31,456 deaths \cite{JHU}.  From this data, hypotheses can be constructed about the progression of the illness.

Using the Johns Hopkins time series of US deaths, we can easily reconstruct how many Americans had the coronavirus three weeks prior.  A case fatality rate of 1.54 percent is used, which comes from cases on cruise ships where 100 percent of passengers were tested \cite{ships}.  This level is obviously an estimate that leads to the largest source of model uncertainty, so the Shiny application developed allows for customizable user input to adjust to more certain rates. In the effect of overwhelmed healthcare systems and intensive care resources, the fatality rate would likely rise much higher.  Three weeks was defined as the time delay between contracting the infection and death.  This is consistent with the median 6 days to symptoms and 13 days to death cited from the Center of Disease Control \cite{cdc}.

A lower fatality rate is a bit of a double-edged sword in this analysis.  If the true fatality rate was 0.1 percent, then this analysis would show 10 times as many calculated infected people.  This model is simplistic and does not account for people being cured and developing antibody immunity, the effects of social distancing and lockdown, and innacuracies in fatality rate reporting.  Nonetheless, the model shows an estimated 385,700 new daily cases for March 28, 2020, versus the actual report of 19,733 \cite{JHU}.  This is a dramatic divergence, but relatively consistent with reports stating that the actual cases represented just 6 percent of infections on average across 40 national governments at the end of March \cite{reason}.  Applying that 6 percent would estimate the total cases on March 28 to be 328,883.

To model the daily deaths, the data was fit to a 2nd order polynomial regression model, adjusted to per capita numbers.  Due to lack of decreasing daily fatalities in the United States, it was difficult to fit a model that would ever come back down to 0.  Many of the cited models did a far more complicated analysis with SIR differential equations that give more scientific predictions.  Using several countries per capita daily fatality curves, I applied the model to the United States current shape. China data was not used in this analysis, due to questions about its accuracy and sharp decline in fatalities. 
