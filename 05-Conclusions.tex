\section{Conclusions}
\graphicspath{{Figures/}{../Figures/}}


COVID-19 has been a massive challenge for the United States healthcare system.  It is not only causing high mortality, but rippling through the entire economy as companies are shut down, people are out of work, and demand for commodities such as oil are tanking.  It is important to model and forecast the impacts of COVID-19 in order to address the systemic risk in our health systems.  There is no real playbook on what the future looks like with COVID-19, with an incomplete time series and lack of knowledge on a novel virus.

While there are many cited models out there forecasting daily deaths, it is a very tough problem to accurately predict without making many assumptions.  Some models perform better than others, and all face the issue small invalid assumptions spiraling out of control when modeling a virus as infectious as this.   However, having multiple models can help increase understanding of the disease, yielding more useful results for decision-makers. Variables not accounted for include:  under reported deaths, lack of testing to accurately gauge the amount of cases, lack of clarity on immunity periods after infection, and general lack of knowledge on a brand new disease.  It is impossible to know the true hospitalization rate, infection rate, and R\typesubscript{0} without fully knowing the number of undiagnosed cases, which is an almost impossible task in a country with a population of 328 million. Additionally, modeling the disease with symmetric Gaussian-like curves may be not the right approach, as shown in the shapes of daily deaths in Italy and Spain reaching plateaus rather than a sharp decreasing slope.

As the future of this disease unfolds, it will be interesting to see the impacts of social distancing, quarantine, and stay at home orders had on its spread.  Additionally, if everything opens back up and returns to normal, it will be critical to monitor spread in order to limit as few deaths and as much strain as possible on our healthcare system.  The dynamic model developed in this analysis can adapt to different assumptions, and hopefully provide a useful prediction for the United States future.







