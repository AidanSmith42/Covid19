\section{Introduction}
\label{Intro}

In early December 2019, the novel coronavirus (severe acute respiratory syndrome [SARS]- COV-2, also known as COVID-19) was first identified from patients with severe pnemonia-like symptoms \cite{long} .  In the next several months, this Coronavirus spread globally and turned into a worldwide pandemic.  The spread and impact of this novel virus has been a major problem globally, but the United States has been greatly effected.  This paper aims to address and analyze the impact of COVID-19 in the United States, as well as discuss several of the many models that attempt to forecast fatality impacts.  

    COVID-19 is not only causing mortality, but also comes with large systemic risk to the healthcare system.  There has been many comparisons to the flu by the media, however COVID-19 brings significantly more risk due to the unpredictability and lack of a vaccine.  As we have seen in Italy and many other countries, healthcare systems have been overrun with COVID-19 cases.  This results in triage measures due to lack of hospital beds, ventilators, and available staff, which greatly inflates the fatality rate and overall fatalities.  Accurately modeling this risk is paramount in order to accurately forecast the demand for these Intensive Care health resources in order to minimize the death toll of this virus.
    
    Learning more about this disease and its attributes is essential for any modeling to be done.  In epidemiology, R\textsubscript{0} is a mathematical term that indicates how contagious an infectious disease is.  R\textsubscript{0} tells you the average number of people who will catch a disease from one contagious person.  COVID-19 has an estimated R\textsubscript{0} of 2.4 \cite{who}.  Additionally, variables such as the amount of confirmed cases, and the time delay between infection and death are critical to forecasting the impact on the health system.  Unfortunately due to lack of mass testing, the amount of confirmed cases is merely a function of the amount of tests being performed.  Thus, to get a good estimate, the total case population must be calculated to get a more accurate number.
    
    There have been several cited models touted by President Trump and the Coronavirus Task Force.  The one most widely used is from the Institute for Health Metrics and Evaluation (IHME) \cite{imhe}.  However, this model seems to considerably under-predict projected fatalities.  Similarly, the CovidActNow model seems to over-project fatalities by making poor assumptions on reproductive number of infections\cite{covidactnow}.  The Los Alamos group model is the best one out there that is close to the actual numbers \cite{losalamos}.  However, all models are likely flawed as this is a very tough problem to solve and model. 

%\vskip .01in
%\includegraphics[scale=.125]{Figures/Net_Illustration (1).jpg}  
%\caption{ A wireless sensor network}
